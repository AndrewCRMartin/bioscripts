\documentclass{article}
\usepackage{a4}
\usepackage{tabularx}
\title{FTPMirror}
\author{Dr.\ Andrew C.R.\ Martin}

\begin{document}
\maketitle

\section{Introduction}
FTPMirror is a simple program to mirror one or more FTP sites and to
handle compression or decompression of the files. The Sunsite Mirror
script (and indeed the Perl LWP package) seems to fail on very large
remote directories (e.g.\ the PDB which has $>100,000$ files). This script
will handle such large directories.

FTPMirror uses a single configuration file for all the sites to be
mirrored. It can mirror single files or directories and can handle
directory trees if required.

Note that if the remote `file' is actually a symbolic link, by default
ftpmirror will create a local symbolic link instead of an actual
file. Using the \verb|wget| mode will overrride that.

\section{Installation}
Note that you must have the following software installed to use
FTPMirror:

\begin{center}
\begin{tabularx}{\linewidth}{lX}\hline
Perl & the Perl interpreter \\
Perl LWP &  the LWP package for Perl (plus any pre-requisites
that it may need) \\
\verb|wget| &  The \verb|wget| program for downloading files by FTP or
HTTP. Only required if you are mirroring very large directories
using \verb|wget| mode.\\\hline
\end{tabularx}
\end{center}

Using Fedora or CentOS, you can install these using:
\begin{verbatim}
yum -y install perl-LWP-UserAgent-Determined
yum -y install perl-Try-Tiny
yum -y install wget
\end{verbatim}

Once you have downloaded the FTPMirror Perl script, simply place it
somewhere in your path and ensure it is executable:
\begin{verbatim}
chmod +x ftpmirror.pl
\end{verbatim}

\section{Running FTPMirror}
FTPMirror is run simply by typing the command:
\begin{verbatim}
ftpmirror.pl configfile
\end{verbatim}
where configfile is a configuration file as described below.
\vspace{2em}

You can run the program with a \verb|-h| flag to obtain help:
\begin{verbatim}
ftpmirror.pl -h
\end{verbatim}
and with debugging options:
\begin{verbatim}
ftpmirror.pl -debug=n configfile
\end{verbatim}
The \verb|-debug| flag may be set to values 1, 2, or 3 with the
following effects:
\begin{enumerate}
\item Prints results of parsing the configuration file, \verb|wget| reports
  progress 
\item Mirroring exits after 10 files
\item When cleaning local files, the first 10 files from the remote
  file and directory lists are printed
\end{enumerate}
\vspace{2em}

You can also run the program with a \verb|-quiet| flag to suppress all
output about progress of what files are being downloaded or removed.

As of V1.4 (19.05.14), the default behaviour is not to delete any
local files if more than 50\% of files have gone away on the remote
compared with the current local copy. This is designed to prevent
problems with the connection dropping while getting the remote
directory listing. The \verb|-forcedelete| option overrides this behaviour 
for all mirrors and deletes the files anyway. You can also use
\verb|forcedelete| option in the config file for an individual mirror
if you know it is very dynamic and often deletes a lot of files.


\section{The configuration file}
The configuration file consists of lines with two compulsory fields
(the source and destination) and optional fields containing flags as
described below. {\bfseries All fields must appear on a single line.}

The configuration file may contain blank lines and comments introduced
with a hash character(\#).
\vspace{2em}

The first field is the source URL in the form:
\begin{verbatim}
ftp://server/directory/
\end{verbatim}
For example:
\begin{verbatim}
ftp://ftp.wwpdb.org/pub/pdb/data/structures/all/pdb/
\end{verbatim}

The second field contains the full path to the destination directory
in which the mirror will be stored:
\begin{verbatim}
/path/to/mirror/
\end{verbatim}
For example:
\begin{verbatim}
/data/pdb/
\end{verbatim}
This may also be a filename (rather than a directory name) if the
source URL is a file rather than a directory and the `file' flag is
used (see below).

Remaining fields are flags:
\begin{description}
\item[recurse]    recurse into lower directories. By default only the
  current directory specified by the URL will be mirrored. Use this
  option to mirror sub-directories as well.
\item[decompress] decompress remote compressed files. Currently this
  only works with gzipped remote files.
\item[compress]   compress remote uncompressed files. Currently this
  only uses gzip to compress the local files.
\item[file]       this specifies that the remote URL refers to a
  single file rather than a directory
\item[wget]       use \verb|wget| rather than LWP to obtain the remote
  directory listing. This is used for big directories where LWP seems
  to fail. It is also used if the remote `files' are actually symbolic
  links; using \verb|wget| will force the actual files to be downloaded.
\item[retry=n]    this is only valid when \verb|wget| is used and specifies
                  the number of \verb|wget| retries (Default is 1).
                  A value of 0 will keep trying indefinitely.
\item[noclean]    do not clean up local files that have gone away on the
                  remote machine
\item[forcedelete] As of V1.4 (19.05.14), the default behaviour is not
                   to delete any local files if more than 50\% of
                   files have gone away on the remote compared with
                   the current local copy. This is designed to prevent
                   problems with the connection dropping while getting
                   the remote directory listing. The forcedelete
                   option overrides this behaviour and deletes the
                   files anyway. You can also use \verb|-forcedelete|
                   on the command line to achieve the same thing for
                   all mirrors.
\item[fast] just checks if files have appeared/disappeared rather than
                  checking the date-stamps on the files. This means
                  that if the content of a file has changed, that will
                  not be reflected in the mirror. The default is to
                  check the files more carefully by comparing date
                  stamps.
\item[regex=r]    only downloads files if the filename (excluding
                  the path) matches the specified Perl regular
                  expression. This is a full Perl regular expression,
                  so may contain alternatives, anchors, etc.
\item[excl=r]     skip files if the full URL filename path matches the
                  specified Perl regular expression. This is a full
                  Perl regular expression, so may contain
                  alternatives, anchors, etc. This overrides matching
                  with regex= thus allowing you to retrieve all files
                  matching a certain pattern unless they are in 
                  directories which match another pattern
                  
\end{description}

\section{Examples}
The following examples all represent individual lines in the
 configuration file. {\bfseries Each should be entered on a single
 line in the config file, even if broken across multiple lines here.}
\vspace{1em}

\noindent Mirror the PDB:
{\footnotesize
\begin{verbatim}
ftp://ftp.wwpdb.org/pub/pdb/data/structures/all/pdb/ /acrm/data/pdb/ wget
\end{verbatim}
}
This would be slow since it is comparing the date stamps on each
file.
\vspace{1em}

\noindent Mirror the PDB, but don't check the date stamps to speed things up:
{\footnotesize
\begin{verbatim}
ftp://ftp.wwpdb.org/pub/pdb/data/structures/all/pdb/ /acrm/data/pdb/ wget fast
\end{verbatim}
}
\vspace{1em}

\noindent Mirror the PDB, decompressing each local file:
{\footnotesize
\begin{verbatim}
ftp://ftp.wwpdb.org/pub/pdb/data/structures/all/pdb/ /acrm/data/pdb/ wget fast 
      decompress
\end{verbatim}
}
\vspace{1em}

\noindent Mirror the Human SubXXXX.bcp.gz files from dbSNP
{\footnotesize
\begin{verbatim}
ftp://ftp.ncbi.nlm.nih.gov/snp/organisms/human_9606/database/organism_data/ 
      /tmp/mirror/ regex=^Sub.*\.bcp\.gz
\end{verbatim}
}
\vspace{1em}

\noindent Mirror the Human SubXXXX.bcp.gz files from dbSNP but exclude
those that are in archive directories
{\footnotesize
\begin{verbatim}
ftp://ftp.ncbi.nlm.nih.gov/snp/organisms/human_9606/ /tmp/mirror/ 
      regex=^Sub.*\.bcp\.gz excl=archive recurse
\end{verbatim}
}

\noindent Mirror and decompress the patent files from EMBL. These
`files' are actually symbolic links so we use wget mode:
{\footnotesize
\begin{verbatim}
ftp://ftp.ebi.ac.uk/pub/databases/embl/patent/ /tmp/mirror/ decompress fast wget
\end{verbatim}
}

\end{document}


